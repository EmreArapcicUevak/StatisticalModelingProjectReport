\documentclass[a4paper, 10pt]{article}

\usepackage[margin = 1in]{geometry} % for spacing around
\usepackage{tabularx}
\usepackage{graphicx} % for including images in your pdfs
\usepackage{xcolor} % for including colors in your pdf
\usepackage{soul} % for text decoration
\usepackage[utf8]{inputenc} % for encoded text
\usepackage[T1]{fontenc}
\usepackage{setspace} % for setting different line spacings between paragrafs.
\usepackage{enumerate} % for letting us get more detailed enumerate lists
\usepackage{multirow} % to let us combine more rows together
\usepackage{colortbl} % for decorating tables
\usepackage{amsmath} % used for representing more complicated math displays
\usepackage{supertabular}
\usepackage{longtable} % both of these packages are used to making really big tables
\usepackage{wrapfig} % allows us to wrap text around figures
\usepackage{fancyhdr} % for making fancy headers
%\usepackage{bibtex} % for making better bibliographies
\usepackage[pdftex]{hyperref} % for letting us make links
\usepackage{lscape} % Allows us to flip from portrait to landspace
\usepackage{tikz} % for high detailed drawing
\usepackage{multicol} % To put things side by side
\usepackage{rotating} % For rotating objects
% \usepackage{draftwatermark} % For adding watermarks
\usepackage{MnSymbol} % for using multiple symbols
\usepackage{mathtools} % Used for more math symbols
\usepackage{xfrac} % For more complciated fractions and to add derivitives
\usepackage{hyperref} % for hyper links
\usepackage{enumitem} % for better enum lists
\usepackage{tcolorbox} % for adding colored text boxes
\usepackage{bm} % Adding bold text to math inputs
\usepackage{pgfplots} % Used for plotting functions
\usepackage{pgfplotstable} % Used for plotting tables
\usepackage{booktabs} % for nicer tables
\usepackage{longtable} % for long tables that can span multiple pages
\usepackage[toc,page]{appendix} % for adding appendices
\usepackage{array}
\usepackage{colortbl}
\definecolor{lightgray}{gray}{0.95}
\renewcommand{\arraystretch}{1.3}

% Setting up the default image path
\graphicspath{{./Images/}}

% Implementing authro details
\title{}
\author{}
\date{}

% Setting up the fancy page style
\fancypagestyle{customStyle}{
	\lhead{} \chead{} \rhead{}
	\lfoot{} \cfoot{\thepage} \rfoot{}
	\renewcommand{\headrulewidth}{0pt}
	\renewcommand{\footrulewidth}{1pt}
}
\pagestyle{customStyle}

% Setting up hyperref options
\hypersetup {
	colorlinks = false,
	citecolor = black,
	filecolor = blue,
	linkcolor = blue,
	urlcolor = blue,
	pdftex
}

% Custom commands

% Custom column colors
\definecolor{lightblue}{RGB}{220,230,255}
\definecolor{lightgreen}{RGB}{220,255,220}
\definecolor{lightyellow}{RGB}{255,250,210}
\definecolor{lightpink}{RGB}{255,230,230}

\begin{document}
	\begin{titlepage}
    \begin{center}
        \includegraphics[width=0.6\textwidth]{IUS_Logo.png} \\[1.5cm]

        {\Huge \textbf{A Statistical Analysis of Public Tram Transportation in Sarajevo}} \\[1.2cm]

        {\LARGE \textbf{Statistical Modeling}} \\[0.5cm]
        {\Large MATH 306} \\[2cm]

        {\Large \textbf{International University of Sarajevo}} \\[0.3cm]
        {\large \today}
    \end{center}

		\vfill

		\begin{minipage}[t]{0.4\textwidth}
			\begin{flushleft} \large
				\textbf{Students:} \\
				\begin{itemize}
					\item Sanjin Ruzic 
					\item Emre Arapcic-Uevak 
				\end{itemize}
			\end{flushleft}
		\end{minipage}
		\hfill
		\begin{minipage}[t]{0.4\textwidth}
			\begin{flushright} \large
				\textbf{Professor:} \\
				\vspace{3mm}
				Dr. Ozge Buyukdagli
			\end{flushright}
		\end{minipage}
	\end{titlepage}
	\pagebreak
	
	\tableofcontents
	\pagebreak
	
	\section{Introduction and Motivation}
		Our project focuses on performing a statistical analysis to evaluate the accuracy of the public transportation company GRAS's claim that the 
		\emph{average tram wait time} is \textbf{4 minutes}, including during rush hours. \\

		\noindent Based on personal experience and public perception, this figure appeared questionable, especially during peak traffic times, 
		which often feel significantly longer. Therefore, we set out to verify whether this 4-minute average holds true under different conditions. \\

		\noindent In addition to wait times, we were also interested in analyzing the proportion of old vs. new trams operating during rush hours, 
		non-rush hours, and overall. \\
		
		\noindent This distinction is important because older trams are generally not accessible 
		to people in wheelchairs, with the exception of a single retrofitted model known as the “JUMBO TRAM”, 
		which is \emph{rarely} seen in service. By examining the deployment of accessible (\emph{new}) trams across different time periods, we aim to assess how inclusive the current tram system is for passengers with mobility impairments.

	\section{Defined Hypotheses}
		\subsection{Proportion of Old vs. New Trams During Rush Hour}
			\begin{itemize}
					\item \textbf{Null Hypothesis ($H_0$):} The proportion of old and new trams is the same during rush and non-rush hours. ($p_{\text{old, rush}} = p_{\text{old, non-rush}}$)
					\item \textbf{Alternative Hypothesis ($H_a$):} The proportion of old trams is higher during rush hours. ($p_{\text{old, rush}} > p_{\text{old, non-rush}}$)
			\end{itemize}

		\subsection{Average Tram Wait Time (Overall)}
			\begin{itemize}
					\item \textbf{Null Hypothesis ($H_0$):} The average tram wait time, in general, is 4 minutes. ($\mu_0 = 4$)
					\item \textbf{Alternative Hypothesis ($H_a$):} The average tram wait time is greater than 4 minutes. ($\mu_a > 4$)
			\end{itemize}

		\subsection{Average Wait Time: Rush vs. Non-Rush Hours}
			\begin{itemize}
					\item \textbf{Null Hypothesis ($H_0$):} The average wait time is the same during rush and non-rush hours. ($\mu_{\text{rush}} = \mu_{\text{non-rush}}$)
					\item \textbf{Alternative Hypothesis ($H_a$):} The average wait time is greater during rush hours. ($\mu_{\text{rush}} > \mu_{\text{non-rush}}$)
			\end{itemize}

	\section{Data Description and Assumptions}

		\begin{spacing}{1.2}
			To collect the data for our analysis, we conducted \textbf{direct observation} at a tram station. 
			\noindent We waited for the \textbf{first tram to arrive} and used its departure as the \textbf{starting point} for our data collection. 
			After the reference tram left, we started a timer and recorded the \textbf{arrival time of the next tram}. 
			This process was repeated to collect multiple data points on tram arrival intervals. \\

			\noindent It is important to note that the \textbf{reference trams} (i.e., the first trams used to start each measurement cycle) were 
			\textbf{not included} in the dataset for average wait time calculations, but they were 
			\textbf{included in the population proportion analysis} (e.g., counting the number of old vs. new trams). \\

			\noindent We did not conduct surveys or use digital scheduling data; instead, all data was manually recorded based on real-time observations.
		\end{spacing}
		\subsection*{Assumptions and Limitations}

		We assume that:
		\begin{itemize}
				\item Tram arrivals are \textbf{independent} of one another.
				\item Our observation window is \textbf{representative} of general traffic conditions.
				\item Our sampling method is \textbf{non-random but unbiased} in timing—we observed whatever trams happened to arrive, without prior selection.
		\end{itemize}

		\noindent However, there are a few potential \textbf{biases and limitations}:
		\begin{itemize}
				\item \textbf{Rush hour vs. non-rush hour} boundaries were estimated based on perception, not official schedules.
				\item \textbf{External conditions} (e.g., traffic, weather, or delays) may have influenced tram arrivals during data collection.
				\item Data was collected \textbf{manually}, introducing possible human error in timing or tram classification.
		\end{itemize}

		\noindent Despite these limitations, the methodology provides a solid foundation for testing the hypotheses regarding wait times and tram accessibility. \\

		\noindent The raw data table containing all measurements is provided in \textbf{Appendix~\ref{sec:rawdata}}.

	\appendix
	\appendixpage 
	\addappheadtotoc
	\section{CSV Dataset Contents}
	\label{sec:rawdata}
	
	\input{Data/StatisticalAnalysisOfTrams.tex}
\end{document}
