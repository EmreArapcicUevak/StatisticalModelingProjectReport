\documentclass[a4paper, 10pt]{article}

\usepackage[margin = 1in]{geometry} % for spacing around
\usepackage{graphicx} % for including images in your pdfs
\usepackage{xcolor} % for including colors in your pdf
\usepackage{soul} % for text decoration
\usepackage[utf8]{inputenc} % for encoded text
\usepackage[T1]{fontenc}
\usepackage{setspace} % for setting different line spacings between paragrafs.
\usepackage{enumerate} % for letting us get more detailed enumerate lists
\usepackage{multirow} % to let us combine more rows together
\usepackage{colortbl} % for decorating tables
\usepackage{amsmath} % used for representing more complicated math displays
\usepackage{supertabular}
\usepackage{longtable} % both of these packages are used to making really big tables
\usepackage{wrapfig} % allows us to wrap text around figures
\usepackage{fancyhdr} % for making fancy headers
%\usepackage{bibtex} % for making better bibliographies
\usepackage[pdftex]{hyperref} % for letting us make links
\usepackage{lscape} % Allows us to flip from portrait to landspace
\usepackage{tikz} % for high detailed drawing
\usepackage{multicol} % To put things side by side
\usepackage{rotating} % For rotating objects
% \usepackage{draftwatermark} % For adding watermarks
\usepackage{MnSymbol} % for using multiple symbols
\usepackage{mathtools} % Used for more math symbols
\usepackage{xfrac} % For more complciated fractions and to add derivitives
\usepackage{hyperref} % for hyper links
\usepackage{enumitem} % for better enum lists
\usepackage{tcolorbox} % for adding colored text boxes
\usepackage{bm} % Adding bold text to math inputs
\usepackage{pgfplots} % Used for plotting functions

% Setting up the default image path
\graphicspath{{./Images/}}

% Implementing authro details
\title{}
\author{}
\date{}

% Setting up the fancy page style
\fancypagestyle{customStyle}{
	\lhead{} \chead{} \rhead{}
	\lfoot{} \cfoot{\thepage} \rfoot{}
	\renewcommand{\headrulewidth}{0pt}
	\renewcommand{\footrulewidth}{1pt}
}
\pagestyle{customStyle}

% Setting up hyperref options
\hypersetup {
	colorlinks = false,
	citecolor = black,
	filecolor = blue,
	linkcolor = blue,
	urlcolor = blue,
	pdftex
}

% Custom commands


\begin{document}
	\begin{titlepage}
    \begin{center}
        \includegraphics[width=0.6\textwidth]{IUS_Logo.png} \\[1.5cm]

        {\Huge \textbf{A Statistical Analysis of Public Tram Transportation in Sarajevo}} \\[1.2cm]

        {\LARGE \textbf{Statistical Modeling}} \\[0.5cm]
        {\Large MATH 306} \\[2cm]

        {\Large \textbf{International University of Sarajevo}} \\[0.3cm]
        {\large \today}
    \end{center}

		\vfill

		\begin{minipage}[t]{0.4\textwidth}
			\begin{flushleft} \large
				\textbf{Students:} \\
				\begin{itemize}
					\item Sanjin Ruzic 
					\item Emre Arapcic-Uevak 
				\end{itemize}
			\end{flushleft}
		\end{minipage}
		\hfill
		\begin{minipage}[t]{0.4\textwidth}
			\begin{flushright} \large
				\textbf{Professor:} \\
				\vspace{3mm}
				Dr. Ozge Buyukdagli
			\end{flushright}
		\end{minipage}
	\end{titlepage}
	\pagebreak
	
	\tableofcontents
	\pagebreak
	
	\section{Introduction and Motivation}
		Our project focuses on performing a statistical analysis to evaluate the accuracy of the public transportation company GRAS's claim that the 
		\emph{average tram wait time} is \textbf{4 minutes}, including during rush hours. \\

		\noindent Based on personal experience and public perception, this figure appeared questionable, especially during peak traffic times, 
		which often feel significantly longer. Therefore, we set out to verify whether this 4-minute average holds true under different conditions. \\

		\noindent In addition to wait times, we were also interested in analyzing the proportion of old vs. new trams operating during rush hours, 
		non-rush hours, and overall. \\
		
		\noindent This distinction is important because older trams are generally not accessible 
		to people in wheelchairs, with the exception of a single retrofitted model known as the “JUMBO TRAM”, 
		which is \emph{rarely} seen in service. By examining the deployment of accessible (\emph{new}) trams across different time periods, we aim to assess how inclusive the current tram system is for passengers with mobility impairments.
\end{document}
